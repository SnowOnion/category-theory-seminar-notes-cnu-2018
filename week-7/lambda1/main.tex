%!TEX program = xelatex
% 注意在 Overleaf 要手动把 project 的 compiler 改成 XeLaTeX
\documentclass{article}
\usepackage[top=0.5in, bottom=1in, left=1.25in, right=1.25in]{geometry} % 页边距
\usepackage{xeCJK} % xeCJK 中文最小模板 via http://seisman.info/mini-template-for-xeCJK.html
\usepackage{amsfonts} % \mathbb
\usepackage{amsmath} % \DeclareMathOperator
\usepackage{ebproof}   % Sequent calculi, easier % Thank you, Yinqiu Zhu!
\usepackage{tabularx} % 临时分栏 http://bbs.ctex.org/forum.php?mod=viewthread&tid=78284
\usepackage{array} % 表格 https://zhuanlan.zhihu.com/p/19749566
\usepackage{datetime2} % 不仅要日期还要时间 https://tex.stackexchange.com/questions/2760/omitting-the-date-when-using-maketitle
\usepackage{enumitem} % 定制列表 https://www.latex-tutorial.com/tutorials/lists/\usepa

% 2018-09-23 LGC
%\usepackage{mathabx} % \Dashv not beautiful! No double direction!
\usepackage{turnstile} % well… 支持\not(?), 不错啊
% http://mirrors.huaweicloud.com/repository/toolkit/CTAN/macros/latex/contrib/turnstile/turnstile-en/turnstile_article.pdf
% https://ctan.org/pkg/turnstile
\def\seq{\sdststile{}{}} % “semantically equivalent”
\def\nseq{\not\sdststile{}{}}

\usepackage{amsthm} % \begin{proof} 证明过程 \end{proof}
\usepackage{mathrsfs}% \mathscr
\usepackage{graphicx} 
%\includegraphics[scale=0.6]{fullscreen.png}
% eg: width=3cm[缩放因子], height=8 cm[缩放因子] scale=0.4[缩放因子]
% 浮动体

%\begin{figure}[ht]
%	\centering
%	\includegraphics[width=13.5cm]{2(b).JPG}
%	\caption{Expected Truth Tables of $I_0,\cdots,I_4$ }
%	\label{fig:proving-imply-cannot-derive-and}
%\end{figure}

\usepackage{hyperref}


% 从 PT 2017 hw5 copy 改造(该模块化了!)

\def\To{\Rightarrow}

\def\a{\alpha}
\def\b{\beta}
\def\c{\gamma}
\def\d{\delta}
\def\C{\Gamma}
\def\D{\Delta}


\def\S{\Sigma}
\def\P{\Pi}

\def\e{\eta}
\def\t{\theta}
\def\T{\Theta}

\def\AX{(AX)}
\def\LB{(L$\bot$)}
\def\LA{(L$\land$)}
\def\LAL{(L$\land_L$)}
\def\LAR{(L$\land_R$)}
\def\RA{(R$\land$)}
\def\LV{(L$\lor$)}
\def\RV{(R$\lor$)}
\def\RVL{(R$\lor_L$)}
\def\RVR{(R$\lor_R$)}

\def\RI{(R$\to$)}
\def\LI{(L$\to$)}

\def\LW{(LW)}
\def\RW{(RW)}
\def\LC{(LC)}
\def\RC{(RC)}

% RC = R Conjunction = R Contraction hmmmmm
% So, RA, RV 2333333
% \def 重定义宏, 没有error没有warning…… 后一个默默覆盖. \newcommand 则会检查, 不能覆盖已有的命令. \renewcommand 则又能覆盖……
% 怎么快速找到哪个命令在哪个宏包被定义的呢?

% 蛤!TeXstudio 的编辑器 也 可以定义宏(快捷键层面)…… shift+F1出个证明环境233

\def\EQ{($\equiv$)}

\def\Gonecp{$ \mathsf{G1cp} $ }
\def\Gthreecp{$ \mathsf{G3cp}$ }
\def\Gthreeip{$ \mathsf{G3ip}$ }
\def\Gthreei{$ \mathsf{G3i}$ }
\def\Gonec{$ \mathsf{G1c} $ }
\def\Gonesfour{$ \mathsf{G1S4} $ } % TODO 为啥最后得加空格?

\def\Goneip{$ \mathsf{G1ip} $ }
\def\Gonei{$ \mathsf{G1i} $ }

\def\Gtwocps{$ \mathsf{G2cp^*} $  }
\def\Gtwomips{$ \mathsf{G2[mi]p^*} $  }
\def\Gtwoips{$ \mathsf{G2ip^*} $  }



\def\LVM{(L$\lor^\otimes$)}

\usepackage{amssymb}

\def\Dia{\Diamond} % diamond is small, Diamond is thin

\def\KBox{(K$\Box$)}
\def\RBox{(R$\Box$)}
\def\FBox{(4$\Box$)}
\def\LBox{(L$\Box$)} % LB = L Bottom... 

\def\LDia{(L$\Dia$)}

\def\LN{(L$\neg$)}
\def\RN{(R$\neg$)}
\def\LNN{(L$\neg\neg$)}
\def\RNN{(R$\neg\neg$)}

\def\LAll{(L$ \forall $)}
\def\LExi{(L$ \exists $)}
\def\RAll{(R$ \forall $)}
\def\RExi{(R$ \exists $)}

%\Gonecp
%
%\AX \LB \LAL \LAR \RA \LV \RVL \RVR \RI \LI
%
%\LW \RW \LC \RC
%
%\EQ



% https://tex.stackexchange.com/questions/78842/nested-enumeration-numbering 嵌套编号 哎这才两层
\renewcommand{\labelenumii}{\theenumii}
\renewcommand{\theenumii}{\theenumi.\arabic{enumii}.}

\def\l{\lambda}



\title{圈论讨论班 - Lambda 演算(1)}
\author{SOnion snowonionlee@gmail.com}
\date{\DTMnow 啊 Overleaf 上默认按零时区时区来的, 如何 fix 东八区?}

\begin{document}
	
\maketitle

\setcounter{section}{-1} % https://www.zhihu.com/question/35598006
\section{Contents TODO 生成目录}


\section{从数学表达式到$\l$表达式}
.

作为开场,本节的内容是$\l$演算在历史上被发明的动机、无类型$\l$演算的语法、以及它的非形式化的语义(也就是``直观理解''). 以便引出后续小节的 $\a\b\c$ 等价规约变换什么的.

本节的许多内容可以在 \verb!王盛颐_让我们谈谈lambda演算.pdf!中找到,因此主要记录讨论班内容提纲(若有余力,会区分计划内容和实际内容). 本文中, 代换、$\b$规约等符号也使用王盛颐pdf里的记号($[N/x]M$表示把$\l$项$M$中自由出现的变量$x$给替换成$\l$项$N$所得到的$\l$项; $\triangleright_{1\b}$是一步$\b$规约,$\triangleright_{\b}$是有限步$\b$规约).


\subsection{无类型$\l$演算的文法}
.

$\l$演算所处理的形式语言的句子被称为$\l$项. 首先我们从(形式上)最简单的无类型$\l$演算入手, 它的$\l$项有三种形式: $M ::= x~|~\l x.M~|~M_1 M_2$.

第一种$x$叫变量, 有无穷多的变量(是可数无穷?), 通常用小写英文字母表示. 第二种$\l x.M$叫$\l$抽象. 第三种$M_1 M_2$叫$\l$应用. 第二、三种里 $x$ 是变量, $M,M_1,M_2$都是$\l$项.

``$\l$表达式''是``$\l$项''的同义词.

\subsection{$\l$抽象:以$x^3+2y$为例}
.

动机: 定义函数.

$f_m(x)=x^3+2y$, $g_m(y)=x^3+2y$, $h_m(x,y)=x^3+2y$ 不一样的.

$f:=\l x.x^3+2y$, $g := \l y.x^3+2y$, $h:=\l x.\l y.x^3+2y$. 其中 $:=$ 是元语言的记号, 不妨称为``绑定''(Haskell 的叫法), 把某个 $\l$项绑定到某名字上.

约定: $\l x$约束的``辖域''尽可能大. 如$\l x.\l y.M N =_{def} \l x.(\l y.(M N))$ 而非 $\l x.(\l y.M) N$ 也非 $(\l x.\l y.M) N$.

讨论班记录: 这个约定引发多次麻烦. 如 (1) 验证 $succ 0$ 等于 $1$ 时; (2)写 $\omega$ 组合子的定义时; (3) 讲Church 1958 年定理($\b$规约终止性判则), 定义 head position 时. 

+ J:有没有唯一可读性定理? 

+ 我认为同时要考虑抽象和应用这两种优先级是挺折腾人的.

+ 差点中途使用 (abs x M) 和 (app M1 M2), 这很垠.\\

不严格地讨论一下 f, g, h 的类型. 

$x\in X,y\in Y,x^3+2y\in Z$ 则 $h : X \to (Y \to Z)$. 对比数学常用的 $h_m : (X\times Y)\to Z$.

柯里化(currying)及其逆映射:

$currying: h_m \mapsto h$, $uncurrying: h \mapsto h_m$.

对集合$A,B$, 范畴论里常把 $A\to B$ 的函数写作 $B^A$(这同时也有避免混淆``从对象A到对象B的态射''和``从集合A到集合B的函数''之功效吧?). 用一下后者符号.

$currying: Z^{X\times Y} \to (Z^Y)^X$. 用范畴语言来看, $X\times Y$ 是两个对象的 product, $Z^Y$ 是 exponential object, currying 是一个 natural isomorphism. (TODO 此处 natural 何解?)

\subsection{$\l$应用}
.

动机: 函数求值、``参数代入''.

约定: $\l$应用是左结合的. 如 $M N P =_{def} (M N) P$.

变量在 $\l$ 项里的自由出现和约束出现. 类比一阶逻辑的量词 $\forall x, \exists x$.

定义: 不含自由变量的 $\l$ 项叫组合子.

\subsubsection{代入的安全性}
.

规定元语言记号$[N/x]M$,表示把$\l$项$M$中自由出现的变量$x$给替换成$\l$项$N$所得到的$\l$项. 

规定: 只有当这个代入安全时才可以这么写.


\subsubsection{$x^3+2y$这东西是$M_1 M_2$的形式吗?}
.

它可以是. 转换方式不唯一.

add (cube x) (mul 2 y)

add (pow x 3) (mul 2 y)

add cube mul pow 2 3 这些是$\l$演算形式语言里的变量, 还是元语言里的被绑定了$\l$项的名字?

我说不好. 但如果是后者, 可以做到. 一种转换方式是邱奇编码(Church's Encoding)(特定到自然数及其运算则称为邱奇数):

$1 := \l f.\l x.f x$

$2 := \l f.\l x.f (f~x)$ 即应用两次

$0 := \l f.\l x.x$ 即应用零次

...

$succ := $

验证 $succ~0$``做代入后''``等于''$1$.

加法、乘法、幂

\section{$\a,\b,\eta$ $\times$ 等价、规约、变换}

所谓``演算''就是字符串变换规则嘛.

\subsection{$\a$ 变换、$\a$等价}

约束变量换名.

不唯一.

\subsection{$\b$ 规约}
.

$\b$ 可约式($\b$-redex).

对$\l$项 P 做一步 $\b$ 规约: 把 P 中出现的 $\b$ 可约式 $(\l x.M)N$ 的一次出现给替换成 $[N/x]M$.

举例

(1)每步唯一且能终止的

(2)会不变短的


\subsubsection{$\b$ 范式($\b$ normal form)}
.

不含$\b$ 可约式的 $\l$ 项.

prop:

\subsubsection{$\b$ 规约的第(3)种情形: 有多种选择}

Thm(Church-Rosser)

term rewriting system; confluence

Thm

Thm (1958 Church)

正则序 应用序

Haskell 的求值规则 = 正则序 + 记忆化

Thm 不可判定

\subsection{$\eta$ 规约、$\eta$ 等价}

$M$ 与 $\l x.M x$.


\section{简单类型 $\l$ 演算}
.

主要参考: Basic Proof Theory 第二版 (A.S.Troelstra 等). 1.2 节 Simple type theories.

讨论班记录: 应听众呼声, 这里先把 \url{https://en.wikipedia.org/wiki/Lambda_cube}画出来了. 本来想晚点画作为结尾. 解释三个方向.

\subsection{简单类型}

基本类型 $A,B,\cdots$; 函数类型 $T_1\to T_2,\cdots$.

\subsection{简单类型 $\l$ 演算(记作$\l_\to$)的$\l$项}

\subsection{$\a,\b,\eta$ $\times$ 等价、规约、变换}

与无类型的定义相同.

\subsection{演算规则}

``该演算的内定理是可规约关系和等价关系''.

这里我自己也没弄清楚. TODO

\section{自然演绎、柯里-霍华德对应}

\subsection{自然演绎}
.

公理系统、矢列演算、自然演绎是三大类演算系统(就是把语义扔一边去只关心语形证明的系统).

(\url{https://sonion.xyz/logic/proof-theory/yu/2017/})

自然演绎里做推导的三种操作: 假设、使用 ($\to I$) 之外的规则、使用($\to I$) 规则并关闭假设.

演示: 证 $\a\to(\b\to\a)$, $\a\to\a$,$(\a\to(\b\to\c))\to((\a\to\b)\to(\a\to\c)$.

\subsection{柯里-霍华德对应}

这里说这个对应的``一个分支''.


\subsection{更多的类型}
.

参考 Awodey 书 2.5 Examples of products

在简单类型 $\l$ 演算的基础上加一种 type operator(具体地, 是 type constructor): 积类型(product type). 在编程语言(Haskell)里对应二元组.

相当于走到了 lambda cube 的 $\l \underline{\omega}$. 当然$\l \underline{\omega}$也有不同丰富程度的类型系统. 比如可以再加和类型(sum type). 在 Haskell 里对应一个类型的多个 data constructor.

CH 对应到合取.

\section{Haskell 演示}

i,k,s 组合子的类型.

$\omega$ 组合子只能在无类型$l$演算中定义. 在 Haskell 里定义不了.

我: Haskell 有类型构造器, 也有多态, 所以类型系统至少是$\l \omega$的, 即$\l \underline{\omega}$``加上''$\l 2$. 但为什么维基百科说 Haskell 的类型系统(HM)是$\l_2$(即 System F)的削弱版, 而不说是 $\l \omega$ 的削弱版?...

预告: Y 组合子, 一种不动点组合子, 用于在``匿名的'' $\l$ 演算里实现函数的递归定义(没名字怎么 call 自己嘛!).


\section{第二次 2018-11-17 Sat}


SKI组合子复习;介绍Y组合子

矢列演算和它的CH对应物(如果你不讲的话)

更多的Haskell演示

\section*{瞎鸡儿分享一番}

\LaTeX 符号手写识别 \url{http://detexify.kirelabs.org/classify.html}

\section*{E.X.2. 文字下脚料}
.

明天大概是80分钟的 与范畴无直接关系的lambda演算+Haskell+柯里霍华德对应, 
和30分钟的一点点范畴论(product那块儿)

\begin{enumerate}
	\item 无类型 lambda 演算
	
	许多内容可以在 \verb!王盛颐_让我们谈谈lambda演算.pdf!中找到.
	\begin{enumerate}
		\item “它用于表示函数”的想法. 
		\begin{enumerate}
			\item 啊啊啊
		\end{enumerate}
	\end{enumerate}
	
	
	\item 简单类型 lambda 演算
	
	约为 Awodey 书 2.5 Examples of products 的前半, 即 p43-45、pdf60-62.
\end{enumerate}	


\begin{verbatim}


1.1. Simply typed lambda calculus 的记号;
1.2. alpha 等价、beta 规约、eta 变换与 currying/uncurrying. 
1.2.1. 几个组合子(不含自由变量的 lambda term)
1.2.2. 运行 Haskell 程序展示上述概念. 

2. 柯里-霍华德对应
(约为 Awodey 书 2.5 Examples of products 的后半, 即 p46、pdf63)
2.1 自然演绎简介
2.2. (极小?直觉主义?)命题逻辑的{蕴含,合取}片段的自然演绎 与 simply typed lambda calculus 的对应. “叫‘同构’暂时不合适, 目前它只是 证明的范畴→类型的范畴 的函子”. 
2.2.1. 运行 Haskell 程序——特别是利用 GHCi 能够显示 lambda term 的类型的能力——展示柯霍对应. 

3. 
(约为 Awodey 书 6.6 λ-calculus(p135 起)和 6.7 Variable sets 的内容, 及其必要的前置知识)
TODO. 草稿:

(chapter 6.6.)
1) cartesian closed poset with finite joins(即海廷代数)与直觉主义命题逻辑的对应(同构?)
2) 上述对应, 是 CCC 与 lambda calculus 的对应, 在 poset 的特殊情况
3) Def:一个 lambda 演算的_理论_ L. 
4) Def:_built from 一个理论 L 的笛卡尔闭范畴_ C(L). 
5) Def(6.16.):L 在范畴 C 里的 _模型_. (lambda 演算的指称语义(之一?))
6) Prop(6.17.):lambda 演算的理论 L 的两个性质

(chapter 6.7.)
1) We conclude with a special kind of CCC related to the so-called Kripke models of logic, namely categories of variable sets. These categories provide specific examples of the “algebraic” semantics of IPC and λ-calculus just given.
一次肯定讲不到这儿吧( ゚∀. ) 先不想了……    
\end{verbatim}
	
\end{document}
