\documentclass[]{article}

\usepackage{xeCJK} % 中文
\usepackage{datetime2} % 不仅要日期还要时间 https://tex.stackexchange.com/questions/2760/omitting-the-date-when-using-maketitle

\usepackage[top=1.0in, bottom=1in, left=1.25in, right=1.25in]{geometry} % 页边距
\usepackage{float} % 然后所有figure的参数都换成H(大写的), 禁止浮动, 文中放哪就放那 https://www.zhihu.com/question/25082703/answer/101218178 

\usepackage{forest}
\usepackage{ifthen} % https://ctan.org/pkg/ifthen

\usepackage{makecell} % forced line break in table. Only used part of the answer https://tex.stackexchange.com/questions/2441/how-to-add-a-forced-line-break-inside-a-table-cell


\def\a{\alpha}
\def\b{\beta}
\def\l{\lambda}

%opening
\title{Precisely Define \\ Head Position and Normal Order Reduction of Lambda Term \\ with Abstract Syntax Tree Technique}
\author{SOnion snowonionlee@gmail.com}
\date{\DTMnow 啊 Overleaf 上默认按零时区时区来的, 如何 fix 东八区?}

\begin{document}

\maketitle

\section{1/2. An AST for Terms of Untyped Lambda Calculus}

This AST (Abstract Syntax Tree) of $\l$-term has 3 kinds s internal nodes: Var (1 child), Abs (2 children), App (2 children). It has 1 kind of leaf node: name of variable.

Here're the examples of correspondence:

\newcommand\ltree[1]{ % 1是参数数量……最多9默认0。#1~#9使用参数。
	\ifthenelse{#1 = 1}{
		\begin{forest}
			[Abs
				[Var[$ x $]]
				[Abs[$ y $][$ M $]]
			]
		\end{forest}
	}{
		\ifthenelse{#1 = 2}{
			\begin{forest}
				[App
					[App[$ M $][$ N $]]
					[$ P $]
				]
			\end{forest}
		}{
			\ifthenelse{#1 = 3}{
				\begin{forest}
					[App
						[Var[$ x $]]
						[App[$ M $][$ N $]]
					]
				\end{forest}
			}{
				\ifthenelse{#1 = 4}{
					\begin{forest}
						[App
							[Abs[Var[$ x $]][$ M $]]
							[$ N $]
						]
					\end{forest}
				}{
					ERROR in document source file!	
				}		
			}
		}
	}
}

% 从给出 makecell 的stackexchange问答贴过来的,没琢磨,反正能让第三行好看(不这么设置就会上下顶格)
\renewcommand\theadalign{bc}
\renewcommand\theadfont{\bfseries}
\renewcommand\theadgape{\Gape[4pt]}
\renewcommand\cellgape{\Gape[4pt]}

% 一开始还用生成器……后来就手写了(一个崩溃点:写\ltreeOne发现两家的生成器都会给 \ 转义;后为路径依赖;有逆向的吗?)
\begin{table}[H]
	\centering
	\begin{tabular}{|c|c|c|} % c 和 l 区别?额好像是居中和左对齐 % TODO 怎么上下紧凑些?草,都包进 makecell 就行,类型是人类的好朋友
		\hline
		ID & $\l$ Term     & AST   \\ \hline
		1 & \makecell{$\l x.(\l y.M)$ \\\\ or $\l x.\l y.M$  because \\ $\l$-abstraction  \\ is right-associative  }  & \makecell{\ltree{1}} \\ \hline
		2 & \makecell{$(M N) P$ \\\\ or $M N P$ because \\ $\l$-application  \\ is left-associative  } & \makecell{\ltree{2}} \\ \hline
		3 & \makecell{$\l x.(M N)$  \\\\ or $\l x.M N$ because \\ $\l$-application has \\ higher precedence \\ than $\l$-abstraction} & \makecell{\ltree{3}}    \\ \hline
		4 & \makecell {$(\l x.M) N$ \\\\ which is the \\ $\b$-redex definition}   & \makecell{\ltree{4}}    \\ \hline
	\end{tabular}
	\caption{Untyped $\l$-Term AST Example}
	\label{table:lambda-term-ast-eg}
\end{table}


\clearpage

\section{2/2. Define Head Position and Normal Order Reduction with AST}

\begin{figure}[H]
	\centering
	\begin{forest}
		[Abs
			[Var[$x_1$]]
			[\dots
				[,phantom]
				[Abs
					[Var[$x_n$]]
					[App
						[\dots
							[App
								[$Q$]
								[$M_1$]]
							[,phantom]]
						[$M_m$]]]]]
	\end{forest}
	\caption{The General AST Form of Untyped $\l$-Term. $n \geq 0, m \geq 0$}
	\label{figure:lambda-term-general-ast-form}
\end{figure}

Suppose the AST in figure \ref{figure:lambda-term-general-ast-form} denotes $\l$-term $t$.

(i) If $Q$'s root is Abs and $m \geq 1$, then \begin{forest}
	[App
		[$Q$]
		[$M_1$]]
\end{forest} is defined as the $\b$-redex on the \textbf{head position} of $t$. Denote this redex as $r$.

% TODO
% \emph{head normal form}
% \emph{\textbf{head normal form}} 又斜又粗
(ii) If $Q$'s root is Var, then define that $t$ is in \textbf{head normal form}.\\


Then, here's why $t$ is the general form of $\l$-Term AST:

(a) If $n=m=0$, then $t$ degenerates to Var;

(b) If $n=0, m \geq 1$, then $t$ degenerates to App;

(c) If $n \geq 1, m = 0$, $t$ degenerates to \begin{forest}
	[Abs
		[,phantom]
		[\dots
			[,phantom]
			[Abs
				[,phantom]
				[Var]]]]
\end{forest} form.\\


Last, defining normal order reduction.

(1) As of case (i), $Q$'s root is Abs and $m \geq 1$ so $t$ has head position $\b$-redex $r$. To do \textbf{normal order reduction} on $t$ is to $\b$-reduce $r$.

(2) As of case (ii), to do \textbf{normal order reduction} on $t$ is to do normal order reduction on the non normal form with the least subscript among $M_1,\cdots,M_m$.


%\begin{table}[]
%	\begin{tabular}{|c|c|c|c|}
%		\hline
%		ID & $\l$ Term   & Remark    & AST   \\ \hline
%		1 & \makecell{$\l x.(\l y.M)$ \\ or } & $\l x.\l y.M$  & \ltree{1} \\ \hline
%		2 & $(M N) P$ & or $M N P$ & \ltree{2} \\ \hline
%		3 & $\l x.(M N)$ & or $\l x.M N$ & \ltree{3}    \\ \hline
%		4 & $(\l x.M) N$ & $\b$-redex definition & \ltree{4}    \\ \hline
%	\end{tabular}
%\end{table}

%
%\section{https://math.stackexchange.com/questions/2768395/head-position-in-lambda-calculus}
%
%\begin{verbatim}
%
%I met the same confusion recently, but I cleared it up by precisely defining “head position” with **AST** (Abstract Syntax Tree) of lambda term.
%
%In the question you actually mentioned the term of only **untyped** lambda calculus. So let’s only concern the untyped.
%
%# 1/2. An AST for Terms of Untyped Lambda Calculus
%
%AST is a very common technique in the theory and practice of programming languages. One can systematically learn it in books or courses concerning **compiler**. 
%
%Here we introduce an AST for *terms of untyped lambda calculus (in short, **lambda term**)*   by examples.
%
%The AST has 3 kinds of **internal nodes**: Var, Abs, App. Var, denoting variable, always has 1 child. Abs, denoting lambda abstraction, and App, denoting lambda application, always have 2 children.
%
%Here’s 4 examples corresponding a lambda term to its AST.
%
%\end{verbatim}
%
%
%$\l x.\l y.M$\quad$\longrightarrow$\quad


\end{document}
